\documentclass[colorlinks]{article}

%% \usepackage[utf8]{inputenc} % Required for inputting international characters
%\usepackage{graphicx}
%\usepackage{amssymb, amsthm}
%%\usepackage[framemethod=default]{mdframed}
%%\usepackage{changepage}
%%\usepackage{caption, subcaption}
%\usepackage{tikz}
%\usepackage{tkz-berge}
%\usetikzlibrary{
%  graphs, graphs.standard, graphdrawing, positioning,
%  shadows.blur, shapes.geometric, arrows.meta,
%  math, calc, intersections, decorations.shapes, quotes
%}
%\usegdlibrary{circular}

\usepackage{hyperref}


\usepackage{amsmath, amsfonts, amssymb, amsthm} % For math equations, theorems, symbols, etc
\usepackage{import}
\usepackage{caption, subcaption}
\usepackage{tikz}
\usepackage{tkz-berge}
\usetikzlibrary{
  graphs, graphs.standard, graphdrawing, positioning,
  shadows.blur, shapes.geometric, arrows.meta,
  math, calc, intersections, decorations.shapes, quotes
}

%\DeclareMathOperator*{\argmin}{arg\,min}
\usegdlibrary{circular}
%\makeatletter
%\renewcommand{\maketitle}{%
%  % Fill PDF information.
%  \hypersetup{%
%    pdftitle    = {\@title},%
%    pdfauthor   = {\@shortauthor},%
%    pdfkeywords = {\@keywords}
%  }

%  % Headers and footers.
%  \ifx\@shortauthor\empty%
%    \ifx\@shorttitle\empty%
%    \else
%      \ihead{\@shorttitle}
%    \fi
%  \else
%    \ifx\@shorttitle\empty%
%      \ihead{\@shortauthor}
%    \else
%      \ihead{\@shortauthor:~\@shorttitle}
%    \fi
%  \fi
%
%  \KOMAoptions{headsepline = true, footsepline = true, plainfootsepline = true}
%  \ModifyLayer[addvoffset=-.6ex]{scrheadings.foot.above.line}
%  \ModifyLayer[addvoffset=-.6ex]{plain.scrheadings.foot.above.line}
%
%  % Footline date.
%  \ifx\@shortdate\empty%
%    \ofoot*{\small \ISOToday}
%  \else
%    \ofoot*{\small \@shortdate}
%  \fi
%  \oldmaketitle%
%  % \vspace{-2\baselineskip}
%}
%\makeatother

\newtheorem{theorem}{Theorem}[section]
\newtheorem{definition}[theorem]{Definition}
\newtheorem{example}{Example}[section]
\newtheorem{remark}[theorem]{Remark}
\newtheorem{corollary}[theorem]{Corollary}
\newtheorem{lemma}[theorem]{Lemma}

\begin{document}
%
\title{Power Grid Optimization}
\author{S. Grundel}
\maketitle
% Introduction
\section{ Introduction to Graph Theory}

\subsection{ Some History of Graph Theory and Its Branches}
Graph Theory began with Leonhard Euler in his study of the Bridges of Königsburg problem.
Since Euler solved this very first problem in Graph Theory, the field has exploded, becoming one of the most important areas of applied mathematics we currently study.
Generally speaking, Graph Theory is a branch of Combinatorics but it is closely connected to Applied Mathematics, Optimization Theory and Computer Science.
Graph Theory is cross-disciplinary between Math, Computer Science, Electrical Engineering and Operations Research.
%
Here are some of the subjects within Graph Theory that are of interest to people in these disciplines:
\begin{enumerate}
\item Optimization Problems on Graphs: Problems of optimization on graphs generally treat a graph structure like a road network and attempt to maximize flow along that network while minimizing costs.
There are many classical optimization problems associated to graphs and this field is sometimes considered a sub-discipline within Combinatorial Optimization.
\item Topological Graph Theory: Asks questions about methods of embedding graphs into topological spaces (like \(\mathbb{R}^2\) or on the surface of a torus) so that certain properties are maintained.
For example, the question of planarity asks:
Can a graph be drawn on the plane in such a way so that no two edge cross.
Clearly, the bridges of Königsburg graph had that property, but not all graphs do.
\item Graph Coloring: A question related both to optimization and to planarity asks how many colors does it take to color each vertex (or edge) of a graph so that no two adjacent vertices have the same color.
Attempting to obtain a coloring of a graph has several applications to scheduling and computer science.
\item Analytic Graph Theory: Is the study of randomness and probability applied to graphs.
Random graph theory is a subset of this study.
In it, we assume that a graph is drawn from a probability distribution that returns graphs and we study the properties that certain distributions of graphs have.
\item Algebraic Graph Theory: Is the application of abstract algebra (sometimes associated with matrix groups) to graph theory.
Many interesting results can be proved about graphs when using matrices and other algebraic properties.
\end{enumerate}
%
Obviously this is not a complete list of all the various problems and applications of Graph Theory.

%
\subsection{A Little Note on Network Science}
If this were a real book, I would never be able to add this section, but since these are lecture notes (and supposed to be educational) it is worth talking for a minute about Network Science.
Network Science is one of these interdisciplinary terms that seems to be appearing everywhere and it seems to be used by anyone who is not a formal mathematician or computer scientist to describe his/her work in some application of graph theory.
%
There are two opinions on Network Science that I have heard so far:
(1) this work is all so brilliant, new and exciting and will change the world or
(2) this is as old as the hills and is just a group of physicists reinterpreting classical results in graph theory and mixing in econometrics-style experiments.
%
Reality, I hope, is somewhere in between.
There is a certain amount of redundancy from older work going on in Network Science.
For example, Simon [Sim55] presaged and surpassed some of the work in the seminal Network Science literature [AB00, AB02, BAJB00, BAJ99] and Alderson et al. [Ald08] do correctly point out that there is a misinterpretation behind the mechanisms of network formation, especially man-made networks.
On the other hand, some questions being asked by the Network Science community are new, useful and highly interdisciplinary such as detecting membership or multiple memberships in online communities (see e.g., [FCF11]) or understanding the spread of pathogens on specialized networks [PSV01, GB06].
It will be interesting to see where this interdisciplinary research goes in the long term.
Hopefully, the results will justify the hype currently surrounding the discipline.
Ideally these notes will help you decide what is really novel and exciting and what is just over-hyped nonsense.

%
\subsection{Basic Graph Definitions}
%
\begin{definition}{(Graph)}\label{def:graph}
A graph is a tuple \(G = (V,E)\) where \(V\) is a (finite) set of vertices and \(E\) is a finite collection of edges.
The set \(E\) contains elements from the union of the one and two element subsets of \(V\).
That is, each edge is either a one or two element subset of \(V\).
\end{definition}
%
\begin{definition}{(Self-Loop)}
If \(G = (V,E)\) is a graph and \(v\in V\) and \(e = \{v\}\), then edge \(e\) is called a self-loop.
That is, any edge that is a single element subset of V is called a self-loop.
\end{definition}
%
\begin{definition}{(Vertex Adjacency)}
Let \(G = (V,E)\) be a graph.
Two vertices \(v_1\) and \(v_2\) are said to be adjacent if there exists an edge \(e \in E\) so that \(e = \{v_1, v_2\}\).
A vertex \(v\) is self-adjacent if \( e = \{v\}\) is an element of E.
\end{definition}
%
\begin{definition}{(Edge Adjacency)}
Let \(G = (V, E)\) be a graph.
Two edges \(e_1\) and \(e_2\) are said to be adjacent if there exists a vertex \(v\) so that \(v\) is an element of both \(e_1\) and \(e_2\) (as sets).
An edge \(e\) is said to be adjacent to a vertex \(v\) if \(v\) is an element of \(e\) as a set.
\end{definition}
%
\begin{definition}{(Neighborhood)}
Let \(G = (V,E)\) be a graph and let \(v\in V\).
The neighbors of \(v\) are the set of vertices that are adjacent to \(v\).
Formally:
\[ N(v)=\{u\in V: \exists e\in E\; e=\{u,v\} \mbox{ or } e={u}\}. \]
In some texts, \(N(v)\) is called the open neighborhood of \(v\) while \(N[v] = N(v) \cup \{v\}\) is called the closed neighborhood of \(v\).
\end{definition}

%
\begin{remark}
The difference between the open and closed neighborhood of a vertex can get a bit odd when you have a graph with self-loops.

\end{remark}
%
\begin{example}\label{ex1}
Consider the set of vertices \(V = \{\text{Hannover}, \text{Leipzig}, \text{Berlin}\}\).
The set of edges \(E = \{\{\text{Hannover}, \text{Leipzig}\}, \{\text{Leipzig}, \text{Berlin}\}, \{\text{Berlin},\text{Hannover} \}\}\).
Then the graph \(G = (V, E)\) has three vertices and three edges.
It is usually easier to represent this graphically.
See Figure~\ref{fig:g1} for the visual representation of \(G\).
%
\begin{figure}
\centering
\begin{tikzpicture}
  \graph [%
    simple necklace layout, nodes={ draw, fill=white, ellipse, blur shadow={shadow blur steps=5}, node sep=10mm}
  ] {%
    Hannover -- Leipzig -- Berlin -- Hannover%
  };
\end{tikzpicture}

\caption{\label{fig:g1} Simple Graph Drawing}
\end{figure}
%
In this example, the neighborhood of Vertex 1 is Vertices 2 and 4; and, Vertex 1 is adjacent to these vertices.
\end{example}

%
\begin{definition}{(Degree)}
Let \(G = (V, E)\) be a graph and let \(v\in V\).
The degree of \(v\), written \(\deg(v)\) is the number of non-self-loop edges adjacent to \(v\) plus two times the number of self-loops defined at \(v\).
More formally:
\[ \deg(v) = \|\{ e\in E : \exists u\in V\; e = \{u, v\}\}\|+ 2 \|\{e \in E : e = \{v\}\}\| \]
Here if \(S\) is a set, then \(\|S\|\) is the cardinality of that set.
\end{definition}
%
\begin{remark}
Note that each vertex in the graph in Figure~\ref{fig:g1} has degree 2.
\end{remark}
%
\begin{example}If we replace the edge set in Example~\ref{ex1} with: \(E = \{\{\text{Hannover, Leipzig}\}, \allowbreak \{\text{Leipzig, Berlin}\}, \allowbreak \{\text{Berlin,Hannover}\}, \allowbreak \{\text{Hannover}\}\}\), then the visual representation of the graph includes a loop that starts and ends at Vertex Hannover.
In this example the degree of Vertex Hannover is now 4.
We obtain this by counting the number of non-self-loop edges adjacent to Vertex Hannover (there are 2) and adding two times the number of self-loops at Vertex Hannover (there is 1) to obtain \(2 + 2 \times 1 = 4\).
\end{example}
%

\begin{theorem}\label{thm:deg}
Let \(G = (V, E)\) be a (general) graph then:
\begin{equation}\label{eq:deg}
2|E| = \sum_{v\in V}  \deg(v)
\end{equation}
\end{theorem}
%
\begin{proof}
Consider two vertices \(v_1\) and \(v_2\) in \(V\).
If \(e = \{v_1, v_2\}\) then it contributed $1$ to  \(\sum_{v\in V}  \deg(v)\) for both \(v_1\) and \(v_2\).
Thus every non-self-loop edge contributes \(2\) to the vertex degree sum.
On the other hand, if \(e = \{v_1\}\) is a self-loop, then this edge contributes \(2\) to the degree of \(v_1\).
Therefore, each edge contributes exactly \(2\) to the vertex degree sum.
Equation~\ref{eq:deg} follows immediately.
\end{proof}
%
\begin{corollary}\label{cor:evenodd}
Let \(G = (V, E)\).
Then there are an even number of vertices in \(V\) with odd degree.
\end{corollary}


\subsection{K\"onigsburg Bridges Example}
\begin{figure}
\centering
\includegraphics[scale=1]{figs/konigsberg_bridges}
\caption{\label{fig:bridge}%
The city of Königsburg is built on a river and consists of four islands, which can be reached by means of seven bridges.
The question Euler was interested in answering is: Is it possible to go from island to island traversing each bridge only once?
(Picture courtesy of Wikipedia and Wikimedia Commons: \url{http://en.wikipedia.org/wiki/File:Konigsberg_bridges.png}).
}%
\end{figure}
%
\begin{figure}
\centering
\def\grlen{2.7}
\begin{tikzpicture}[%
  nodestyle/.style={
    draw, shape=circle, fill=white,
    blur shadow={shadow blur steps=5}
  }
]
  \node [nodestyle] (A) at (0,0) {A};
  \node [nodestyle] (B) at (0,-\grlen) {B};
  \node [nodestyle] (C) at (0,\grlen) {C};
  \node [nodestyle] (D) at (\grlen,0) {D};

  \graph[
    multi,
    edges={semithick}
  ]{
    {(C),(A),(B)} -- (D);
    (C) --[bend left] (A);
    (C) --[bend right] (A);
    (A) --[bend left] (B) ;
    (A) --[bend right] (B);
  };

  \coordinate (labelpos) at ($(C.north east)+(30:2cm)$);
  \draw[dashed,{Stealth[round,length=3mm]}-] plot [smooth] coordinates{
    (A.north west)
    ($(A.north west)+(140:1cm)$)
    ($(C.north west)+(140:1cm)$)
    (labelpos)
  };
  \node[anchor=west] at (labelpos) {Islands};
  \draw[dashed,{Stealth[round,length=3mm]}-]
    (C.north east) -- ($(labelpos)+(0,-1mm)$);
  \draw[dashed,{Stealth[round,length=3mm]}-]
    (D.north) -- ($(labelpos) + (2mm,-1mm)$);
  \draw[dashed, {Stealth[round,length=3mm]}-] plot [smooth] coordinates{
    (B.east)
    ($(B.east)+(0:15mm)$)
    ($(D.south east)+(-45:8mm)$)
    ($(labelpos)+(7mm,-2mm)$)
  };

  \coordinate (labelpos2) at ($(A.south west) + (-140:2cm)$);
  \draw[dashed,-{Stealth[round, length=3mm]}] (labelpos2) -- ++ (0:1.3cm);
  \node[anchor=east] at (labelpos2) {Bridge};
\end{tikzpicture}
\caption{\label{fig:graphbridges}%
Representing each island as a dot and each bridge as a line or curve connecting the dots simplifies the visual representation of the seven Königsburg Bridges.
}%
\end{figure}
%
\begin{example}
The city of Königsburg exists as a collection of islands connected by bridges as shown in Figure~\ref{fig:bridge}.
The problem Euler wanted to analyze was: Is it possible to go from island to island traversing each bridge only once?
This was assuming that there was no trickery such as using a boat.
Euler analyzed the problem by simplifying the representation to a graph.
Assume that we treat each island as a vertex and each bridge as an line egde.
The resulting graph is illustrated in Figure~\ref{fig:graphbridges}.
\end{example}
%
Note this representation dramatically simplifies the analysis of the problem in so far as we can now focus only on the structural properties of this graph.
It is easy to see (from Figure~\ref{fig:graphbridges}) that each vertex has an odd degree.
More importantly, since we are trying to traverse islands without ever recrossing the same bridge (edge), when we enter an island (say C) we will use one of the three edges.
Unless this is our final destination, we must use another edge to leave C.
Additionally, assuming we have not crossed all the bridges yet, we know we must leave C.
That means that the third edge that touches C must be used to return to C a final time.
Alternatively, we could start at Island C and then return once and never come back.
Put simply, our trip around the bridges of Königsburg had better start or end at Island C.
But Islands (vertices) B and D also have this property.
We cannot start and end our travels over the bridges on Islands C, B and D simultaneously- therefore, no such walk around the islands in which we cross each bridge precisely once is possible.

\begin{definition}{(MultiGraph)}
A graph \(G = (V,E)\) is a multigraph if there are two edges \(e_1\) and \(e_2\) in \(E\) so that \(e_1\) and \(e_2\) are equal as sets.
That is, there are two vertices \(v_1\) and \(v_2\) in \(V\)  so that \(e_1=e_2=\{v_1,v_2\}\).
\end{definition}
\begin{remark}
Note in the definition of graph (Definition~\ref{def:graph}) we were very careful to specify that \(E\) is a collection of one and two element subsets of \(V\) rather than to say that \(E\) was, itself, a set.
This allows us to have duplicate edges in the edge set and thus to define multigraphs.
In Computer Science a set that may have duplicate entries is sometimes called a multiset.
A multigraph is a graph in which \(E\) is a multiset.
\end{remark}

Consider the graph associated with the Bridges of Königsburg Problem.
The vertex set is \(V = \{A, B, C, D\}\).
The edge collection is: \(E = \{\{A,B\},\allowbreak\{A,B\},\allowbreak\{A,C\},\allowbreak\{A,C\},\allowbreak\{A,D\},\allowbreak\{B,D\},\allowbreak\{C,D\}\}\).
This multigraph occurs because there are two bridges connecting island \(A\) with island \(B\) and two bridges connecting island \(A\) with island \(C\).
If two vertices are connected by two (or more) edges, then the edges are simply represented as parallel lines (or arcs) connecting the vertices.

%
%\begin{remark}
%Let \(G = (V,E)\) be a graph.
%There are two degree values that are of interest in graph theory: the largest and smallest vertex degrees usually denoted \(\Delta(G)\) and \(\delta(G)\).
%That is:
%\begin{align}
%  \Delta(G) &= \max\deg(v) v'in V \\
%  \delta(G) &= \min\deg(v) v\in V
%\end{align}
%\end{remark}
%
\begin{remark}\label{rem:simple}
Despite our initial investigation of The Bridges of Königsburg Problem as a mechanism for beginning our investigation of graph theory, most of graph theory is not concerned with graphs containing either self-loops or multigraphs.
\end{remark}
%
\begin{definition}{(Simple Graph)}
A graph \(G = (V,E)\) is a simple graph if \(G\) has no edges that are self-loops and if \(E\) is a subset of two element subsets of \(V\); i.e., G is not a multi-graph.
\end{definition}
%
\begin{remark}
In light of Remark~\ref{rem:simple} and the applications in energy networks which will follow, we will assume that every graph we discuss in these notes is a simple graph and we will use the term graph to mean simple graph.
When a particular result holds in a more general setting, we will state it explicitly.
\end{remark}
%\begin{ex} Consider the new Bridges of Ko ̈nigsburg Problem from Exercise 1.
%Is the graph representation of this problem a simple graph? Could a self-loop exist in a graph derived from a Bridges of Ko ̈nigsburg type problem? If so, what would it mean? If not, why?
%\end{ex}


% Graph properties
\section{Graph Properties}


A graph is determined by a set of vertices and edges. In the following we will look at some graph properties and features and some definitions of certain graph types. 



%
%
%\begin{definition}{(Empty and Trivial Graphs)}
%A graph \(G = (V,E)\) in which \(V = \varnothing\) is called the empty graph (or null graph).
%A graph in which \(V = \{v\}\) and \(E = \varnothing\) is called the trivial graph.
%\end{definition}
%\begin{definition}{(Isolated Vertex)}\label{def:isver}
%Let \(G = (V,E)\) be a simple graph and let \(v \in V\).
%If \(\deg(v) = 0\) then v is said to be isolated.
%\end{definition}
%%
%\begin{remark}
%Note that Definition~\ref{def:isver} applies only when \(G\) is a simple graph.
%If \(G\) is a general graph (one with self-loops) then \(v\) is still isolated even when \(\{v\} \in E\), that is there is a self-loop at vertex \(v\) and no other edges are adjacent to \(v\).
%In this case, however, \(\deg(v) = 2\).
%\end{remark}
%%
\begin{definition}{(Degree Sequence)}
Let \(G = (V,E)\) be a graph with \(\|V\| = n\).
The degree sequence of \(G\) is a tuple \(d\in \mathbb{Z}^n\) composed of the degrees of the vertices in \(V\) arranged in decreasing order.
\end{definition}
%
\begin{example}
Consider the graph in Figure~\ref{fig:g3}.
The degrees for the vertices of this graph are:
\[v_1 = 4 \quad v_2 = 3 \quad v_3 = 2 \quad v_4 = 2 \quad v_5 = 1\]
This leads to the degree sequence: \(d = (4, 3, 2, 2, 1)\).
\end{example}
%
\begin{figure}
\centering
\begin{tikzpicture}[nodestyle/.style={draw,shape=circle, fill=white, blur shadow={shadow blur steps=5}},]
  % put nodes at the corners of a pentagon
  \node[regular polygon,regular polygon sides=5,minimum size=3.5cm, shape border rotate=20] (p) {};
  \foreach\x/\y in {1/1,2/4,3/3,4/5,5/2}{
    % name nodes accordingly
    \node[nodestyle] (p\y) at (p.corner \x){\y};
  }
  % draw edges
  \draw (p1) -- (p2);
  \draw (p1) -- (p3);
  \draw (p1) -- (p4);
  \draw (p1) -- (p5);
  \draw (p2) -- (p3);
  \draw (p2) -- (p4);
\end{tikzpicture}
\caption{\label{fig:g3}%
The graph above has a degree sequence \(d = (4; 3; 2; 2; 1)\).
These are the degrees of the vertices in the graph arranged in decreasing order.
}
\end{figure}
%


\begin{remark} (Pigeonhole principle).
Suppose items may be classified according to \(m\) possible types and we are given \(n> m\) items.
Then there are at least two items with the same type.
This Pigeonhole principle was originally formulated by thinking of placing \(m + 1\) pigeons into \(m\) pigeon holes.
Clearly to place all the pigeons in the holes, one hole has two pigeons in it.
The holes are the types (each whole is a different type) and the pigeons are the items.

\end{remark}
%
\begin{theorem}
Let \(G = (V, E)\) be a non-empty, non-trivial graph.
Then \(G\) has at least one pair of vertices with equal degree.
\end{theorem}
%
\begin{proof}
This proof uses the Pigeonhole principle and is illustrated by the graph in Figure~\ref{fig:g3}, where \(\deg(v_3) = \deg(v_4)\).
The types will be the possible vertex degree values and the objects will be the vertices.

Suppose  \(|V| = n\).
Each vertex could have a degree between \(0\) and \(n-1\) (for a total of \(n\) possible degrees), but if the graph has a vertex of degree \(0\) then it cannot have a vertex of degree \(n-1\).
Therefore, there are only at most \(n-1\) possible degree values depending on whether the graph has an isolated vertex or a vertex with degree \(n-1\) (if it has neither, there are even fewer than \(n-1\) possible degree values).
Thus by the Pigeonhole Principle, at least two vertices must have the same degree.
\end{proof}
%
\begin{definition}{(Graphic Sequence)}
Let \(d = (d_1, \ldots, d_n)\) be a tuple in \(\mathbb{Z}^n\) with \(d_1 \geq d_2 \geq \cdots \geq d_n\).
Then \(d\) is graphic if there exists a graph \(G\) with degree sequence \(d\).
\end{definition}
%
\begin{corollary}
If d is graphic, then the sum of its elements is even.
\end{corollary}
%
\begin{proof}
This follows from Theorem~\ref{thm:deg}.
\end{proof}
%


\subsection{Special Graphs}
\begin{definition} {(Complete Graph)}
Let \(G = (V,E)\) be a graph with \(|V| = n\) with \(n \geq 1\).
If the degree sequence of \(G\) is \((n-1,n-1,\dots,n-1) \) then \(G\) is called a complete graph on \(n\) vertices and is denoted \(K_n\).
In a complete graph on \(n\) vertices each vertex is connected to every other vertex by an edge.
\end{definition}

\begin{lemma}
Let \(K_n = (V, E)\) be the complete graph on \(n\) vertices.
Then: \(|E| = \frac{n(n-1)}{2}\).
\end{lemma}
%
\begin{corollary}
Let \(G = (V, E)\) be a graph and let \(|V | = n\).
Then:
\[0\leq |E|\leq  \begin{pmatrix}n\\2\end{pmatrix}.\]
\end{corollary}

\begin{definition}{(Regular Graph)}
Let \(G = (V,E)\) be a graph with \(|V| = n\).
If the degree sequence of \(G\) is \((k, k,\ldots, k)\) with \(k \leq n-1\) then \(G\) is called a \(k\)-regular graph on \(n\) vertices.
\end{definition}
%


%
\begin{figure}
  \centering
  \begin{subfigure}[b]{0.32\textwidth}
    \centering
    \begin{tikzpicture}[baseline, nodestyle/.style={node font=\tiny, shape=circle, fill=yellow}]
  % put nodes at the corners of a triangle
  \node[regular polygon,regular polygon sides=3,minimum size=5cm] (p) {};
  % nodes at corners
  \foreach\x/\y in {1/2,2/4,3/3}{\node[nodestyle] (p\y) at (p.corner \x){\y};}
  % node at centroid
  \node[nodestyle] (p1) at (p.center) {1};
  % draw edges
  \graph[edges=blue]{
    \foreach\x in {1,2,3}{
      \foreach\y in {\x,...,4}{
        (p\x) -- (p\y);
      }
    }
  };
\end{tikzpicture}
    \caption{\(K_4\)}\label{fig:graph4a}
  \end{subfigure}
  \begin{subfigure}[b]{0.32\textwidth}
    \centering
    % Modified from https://tex.stackexchange.com/a/625537
\begin{tikzpicture}[
  baseline,
  every node/.style={node font=\tiny, shape=circle, fill=yellow, inner sep=1pt, minimum size=2pt},
  every edge/.style={draw=blue}
]
  \graph[math nodes, clockwise] {
      subgraph I_n [V={6,9,7,10,8}] --
      subgraph C_n [V={1,2,3,4,5},radius=1.23cm];
      {[cycle] 6,7,8,9,10}
    };
\end{tikzpicture}
    \caption{Petersen Graph}\label{fig:graph4b}
  \end{subfigure}
  \begin{subfigure}[b]{0.32\textwidth}
    \centering
    % Modified from: https://www.integral-domain.org/lwilliams/Resources/TikzImg/dodecagraph.tex

\begin{tikzpicture}[
  baseline,
  every node/.style = {
    circle,
    fill=yellow,
    node font=\tiny,
    inner sep=1pt,
    minimum size=2pt
  },
  uedge/.style={draw=blue}
]
    \foreach \x [
      evaluate=\x as \xo using {int(mod(6-\x,5)+1)},
      evaluate=\x as \xi using {int(mod(6-\x,5)+6)},
      evaluate=\x as \xii using {int(mod(5-\x,5)+11)},
      evaluate=\x as \xiii using {int(mod(5-\x,5)+16)},
    ] in {0,1,2,3,4}{
        \node (o\x) at (18+\x*72:2.2cm) {\xo};
        \node (i\x) at (18+\x*72:1.5cm) {\xi};
        \node (ii\x) at (54+\x*72:1cm) {\xii};
        \node (iii\x) at (54+\x*72:0.5cm) {\xiii};
    }
    \foreach \x in {0,1,2,3,4}{
        \path[uedge] (o\x) edge (i\x);
        \path[uedge] (ii\x) edge (iii\x);
    }
    \path[uedge] (o0)--(o1)--(o2)--(o3)--(o4)--(o0);
    \path[uedge] (iii0)--(iii1)--(iii2)--(iii3)--(iii4)--(iii0);
    \path[uedge] (i0)--(ii0)--(i1)--(ii1)--(i2)--(ii2)--(i3)--(ii3)--(i4)--(ii4)--(i0);
\end{tikzpicture}
    \caption{Dodecahedron}\label{fig:graph4c}
  \end{subfigure}
  \caption{This are three examples of 3-regular graphs, where $K_4$ (a) is a complete graph but the others are not. The Peterson Graph (b) is a 3-regular graph that is used in many graph theoretic examples. The dodecahedron (c) really is a flattened dodecahedron, one of the five platonic solids from classical geometry.}\label{fig:graph4}
\end{figure}
%
\subsection{Directed Graphs}
%
\begin{definition}{(Directed Graph)}
A directed graph (digraph) is a tuple \(G = (V,E)\) where \(V\) is a (finite) set of vertices and \(E\) is a collection of elements contained in \(V \times V\).
That is, \(E\) is a collection of ordered pairs of vertices.
The edges in \(E\) are called directed edges to distinguish them from those edges in Definition~\ref{def:graph}.
\end{definition}
%
\begin{definition}{(Source / Destination)}
Let \(G = (V,E)\) be a directed graph.
The source (or tail) of the (directed) edge \(e = (v_1,v_2)\) is \(v_1\) while the destination (or sink or head) of the edge is \(v_2\).
\end{definition}
%
\begin{remark}
A directed graph (digraph) differs from a graph only insofar as we replace the concept of an edge as a set with the idea that an edge as an ordered pair in which the ordering gives some notion of direction of flow.
In the context of a digraph, a self-loop is an ordered pair with form \((v,v)\).
We can define a multi-digraph if we allow the set \(E\) to be a true collection (rather than a set) that contains multiple copies of an ordered pair.
\end{remark}
%
\begin{remark}
It is worth noting that the ordered pair \((v_1,v_2)\) is distinct from the pair \((v_2, v_1)\).
Thus if a digraph \(G = (V, E)\) has both \((v_1, v_2)\) and \((v_2, v_1)\) in its edge set, it is not a multi-digraph.
\end{remark}
%
\begin{example}
We can modify the figures in Example~\ref{ex1} to make it directed.
Suppose we have the directed graph with vertex set \(V = \{1, 2, 3, 4\}\) and edge set: \(E = \{(1, 2),\allowbreak (2, 3),\allowbreak (3, 4),\allowbreak (4, 1)\}\).
This digraph is visualized in Figure~\ref{fig:gr2-a}.
In drawing a digraph, we simply append arrow- heads to the destination associated with a directed edge.
We can likewise modify our self-loop example to make it directed.
In this case, our edge set becomes: \(E = \{(1, 2),\allowbreak (2, 3),\allowbreak (3, 4),\allowbreak (4, 1),\allowbreak (1, 1)\}\).
This is shown in Figure~\ref{fig:gr2-b}.
\end{example}
%
\begin{figure}
\centering
\begin{subfigure}[b]{0.49\textwidth}
  \centering
  \begin{tikzpicture}[>={Stealth[round]}, semithick]
  \graph [%
    simple necklace layout, nodes={ draw, circle, node sep=2cm}
  ] {%
    1 -> [orient=0]  2 -> 3 -> 4 -> 1%
  };
\end{tikzpicture}
  \caption{\label{fig:gr2-a}}%
\end{subfigure}
\begin{subfigure}[b]{0.49\textwidth}
  \centering
  \tikzset{every loop/.style={in=60,out=120}}
\begin{tikzpicture}[>={Stealth[round]}, semithick]
  \graph [%
    simple necklace layout, nodes={ draw, circle, node sep=2cm}
  ] {%
    1 -> [orient=0]  2 -> 3 -> 4 -> 1 -> [loop left] 1 %
  };
\end{tikzpicture}
  \caption{\label{fig:gr2-b}}%
\end{subfigure}
\caption{
(a) A directed graph.
(b) A directed graph with a self-loop.
In a directed graph, edges are directed; that is they are ordered pairs of elements drawn from the vertex set.
The ordering of the pair gives the direction of the edge.
}
\end{figure}
%
\begin{definition}{(Underlying Graph)}
If \(G = (V,E)\) is a digraph, then the underlying graph of G is the (multi) graph (with self-loops) that results when each directed edge \((v_1, v_2)\) is replaced by the set \(\{v_1, v_2\}\) thus making the edge non-directional.
Naturally if the directed edge is a directed self-loop \((v, v)\) then it is replaced by the singleton set \({v}\).
\end{definition}
%
\begin{remark}
Notions like edge and vertex adjacency and neighborhood can be extended to digraphs by simply defining them with respect to the underlying graph of a digraph.
Thus the neighborhood of a vertex \(v\) in a digraph \(G\) is \(N(v)\) computed in the underlying graph.
\end{remark}
%
\begin{remark}
Whether the underlying graph of a digraph is a multi-graph or not usually has no bearing on relevant properties.
In general, an author will state whether two directed edges \((v_1, v_2\)) and \((v_2, v_1)\) are combined into a single set \(\{v_1, v_2\}\) or two sets in a multiset.
As a rule-of-thumb, multi-digraphs will have underlying multigraphs, while digraphs generally have underlying graphs that are not multi-graphs.
\end{remark}
%
\begin{remark}
It is possible to mix (undirected) edges and directed edges together into a very general definition of a graph with both undirected and directed edges.
Situations requiring such a model almost never occur in modeling and when they do, the undirected edges with form \(\{v_1, v_2\}\) are usually replaced with a pair of directed edges \((v_1, v_2)\) and \((v_2, v_1)\).
Thus, for remainder of these notes, unless otherwise stated:
\begin{enumerate}
\item When we say `graph', we will mean simple graph as in Remark~\ref{rem:simple}.
If we intend the result to apply to any graph we’ll say a general graph.
\item When we say `digraph', we will mean a directed graph \(G = (V, E)\) in which every edge is a directed edge and the component \(E\) is a set and in which there are no self-loops.
\end{enumerate}
\end{remark}
\begin{definition}{(In-Degree, Out-Degree)}
Let \(G = (V, E)\) be a digraph.
The in-degree of a vertex \(v\) in \(G\) is the total number of edges in \(E\) with destination \(v\).
The out-degree of \(v\) is the total number of edges in \(E\) with source \(v\).
We will denote the in-degree of \(v\) by \(\deg_{in}(v)\) and the out-degree by \(\deg_{out}(v)\).
\end{definition}

\begin{theorem}
Let \(G = (V, E)\) be a digraph.
Then the following holds:
\begin{equation}
|E| =\sum_{v\in V} \deg_{in}(v) =\sum_{v\in V} \deg_{out}(v).
\end{equation}
\end{theorem}

\subsection{Subgraphs}
%
\begin{definition}{(Subgraph)}
Let \(G = (V, E)\).
A graph \(H = (V', E')\) is a subgraph of \(G\) if \(V' \subseteq V\) and \(E'\subseteq E\).
The subgraph \(H\) is proper if \(V' \subsetneq V\) or \(E' \subsetneq E\).
\end{definition}
%
\begin{example}%
We illustrate the notion of a sub-graph in Figure~\ref{fig:subg}.
Here we illustrate a sub-graph of the Petersen Graph.
The sub-graph contains vertices 6, 7, 8, 9 and 10 and the edges connecting them.
\end{example}
%
\begin{definition}{(Spanning Subgraph)}
Let \(G = (V, E)\) be a graph and \(H = (V', E') \) be a subgraph of \(G\).
The subgraph \(H\) is a spanning subgraph of \(G\) if \(V' = V\).
\end{definition}

\begin{figure}
  \centering
  \begin{subfigure}{0.32\textwidth}
    \centering
    \input{tikz/graph4b}
    \caption{Petersen Graph}
  \end{subfigure}
  \begin{subfigure}{0.32\textwidth}
    \centering
    % Modified from https://tex.stackexchange.com/a/625537
\begin{tikzpicture}[
  every node/.style={node font=\tiny, shape=circle, fill=yellow},
  every edge/.style={draw=blue}
]
  \graph[math nodes, clockwise, edge=red] {
      subgraph I_n [V={6,9,7,10,8}, nodes={fill=green}] -- [edges=blue]
      subgraph C_n [V={1,2,3,4,5},radius=1.25cm, edge=blue];
      {[cycle] 6,7,8,9,10}
    };
\end{tikzpicture}
    \caption{Highlighted Subgraph}
  \end{subfigure}
  \begin{subfigure}{0.32\textwidth}
    \centering
    \begin{tikzpicture}[
  every node/.style={node font=\tiny, shape=circle, fill=yellow},
  every edge/.style={draw=blue}
]
\graph [clockwise] { subgraph C_n [V={6,...,10}, radius=2.2cm] };
\end{tikzpicture}
    \caption{Extracted Subgraph}
  \end{subfigure}
  \caption{\label{fig:subg}%
The Petersen Graph is shown (a) with a sub-graph highlighted (b) and that sub-graph displayed on its own (c).
A sub-graph of a graph is another graph whose vertices and edges are sub-collections of those of the original graph.
}%
\end{figure}
%
%\begin{definition}{(Edge Induced Subgraph)}
%Let \(G = (V, E)\) be a graph.
%If \(E'\subseteq E\).
%The subgraph of \(G\) induced by \(E'\) is the graph \(H=(V',E')\)
%where \(v\in V'\) if and only if \(v\) appears in an edge in \(E\).
%\end{definition}
%%
%\begin{definition}{ (Vertex Induced Subgraph)}
%Let \(G = (V, E)\) be a graph.
%If \(V'\subseteq V\), the subgraph of \(G\) induced by \(V'\) is the graph \(H=(V',E')\) where \({v_1,v_2}\in E'\) if and only if \(v_1\) and \(v_2\) are both in \(V'\).
%\end{definition}
%
\begin{remark}
For directed graphs, all sub-graph definitions are modified in the obvious way.
Edges become directed as one would expect.
\end{remark}


% Graph definitions and Terminology
\section{Acyclic Graphs (Trees and Forests)}
An acyclic graph has no cycles.
%
\begin{definition}{(Walk)}
Let \(G = (V;E)\) be a graph.
A walk \(w = (v_1; e_1;\allowbreak v_2; e_2; \dots;\allowbreak v_n; e_n; v_{n+1})\) in \(G\) is an alternating sequence of vertices and edges in \(V\) and \(E\) respectively so that for all \(i = 1,\dots, n: \{v_i, v_{i+1}\} = e_i\).
A walk is called closed if \(v_1 = v_{n+1}\) and open otherwise.
A walk consisting of only one vertex is called trivial.
\end{definition}

\begin{remark}
Let \(G = (V,E)\) to each walk \(w = (v_1, e_1,\allowbreak v_2, e_2,\dots,\allowbreak v_n, e_n, v_{n+1})\) we can associated a subgraph \(H = (V_0,E_0)\) with:
\begin{enumerate}
\item  \(V_0 =\{ v_1,\dots,v_{n+1}\}\)
\item  \(E_0 =\{ e_1.\dots,e_n\}\)
\end{enumerate}
We will call this the sub-graph induced by the walk \(w\).
\end{remark}

\begin{definition}{(Length)}
The length of a walk \(w\) is the number of edges contained in it.
\end{definition}
%


\begin{definition}{(Trail)}
Let \(G = (V,E)\) be a graph.
A trail in \(G\) is a walk in which no edge is repeated.
An Eulerian trail is a trail that contains exactly one copy of each edge in \(E\).
\end{definition}

\begin{definition}{(Path)}
Let \(G = (V,E)\) be a graph.
A path in \(G\) is a non-trivial walk with no vertex and no edge repeated.
A Hamiltonian path is a path that contains exactly one copy of each vertex in \(V\).
\end{definition}




\begin{example}
We illustrate a walk, cycle, Eulerian tour and a Hamiltonian path in Figure~\ref{fig:g7}.
A walk is illustrated in Figure~\ref{fig:g7a}.
Formally, this walk can be written as:
\(w = (1, \{1, 4\},\allowbreak 4, \{4, 2\},\allowbreak 2, \{2, 3\},\allowbreak 3)\).
The cycle shown in Figure~\ref{fig:g7b} can be formally written as:
\[c = (1, \{1, 4\},\allowbreak 4,\{ 4, 2\},\allowbreak 2, \{2, 3\},\allowbreak 3,\{ 3, 1\},\allowbreak 1).\]
Notice that the cycle begins and ends with the same vertex (that's what makes it a cycle).
Also, \(w\) is a sub-walk of \(c\). Note further we could easily have represented the walk as:
\[w = (3, \{3, 2\},\allowbreak 2,\{ 2, 4\},\allowbreak 4,\{4,1\},\allowbreak 1).\]
We could have shifted the ordering of the cycle in anyway (for example beginning at vertex 2).
Thus we see that in an undirected graph, a cycle or walk representation may not be unique.
In Figure~\ref{fig:g7c} we illustrate an Eulerian Trail.
This walk contains every edge in the graph with no repeats.
We note that Vertex 1 is repeated in the trail, meaning this is not a path.
We contrast this with Figure~\ref{fig:g7d} which shows a Hamiltonian path.
Here each vertex occurs exactly once in the illustrated path, but not all the edges are included.
In this graph, it is impossible to have either a Hamiltonian Cycle or an Eulerian Tour.
\end{example}
%
\begin{figure}
\centering
\begin{subfigure}{0.24\textwidth}
  \centering
  \input{tikz/walkpath_a}
  \caption{\label{fig:g7a}Walk}
\end{subfigure}
\begin{subfigure}{0.24\textwidth}
  \centering
  \input{tikz/walkpath_b}
  \caption{\label{fig:g7b}Cycle}
\end{subfigure}
\begin{subfigure}{0.24\textwidth}
  \centering
  \input{tikz/walkpath_c}
  \caption{\label{fig:g7c}Eulerian Trail}
\end{subfigure}
\begin{subfigure}{0.24\textwidth}
  \centering
  \begin{tikzpicture}[
  nodestyle/.style={
    draw,
    shape=circle,
    fill=white,
    blur shadow={shadow blur steps=5}
  },
  edgename/.style={
    fill=white,
    shape=circle,
    node font=\tiny
  },
  >={Stealth[round]}
]
  % put nodes at the corners of a pentagon
  \node[regular polygon,regular polygon sides=5,minimum size=2.5cm, shape border rotate=20] (p) {};
  \foreach\x/\y in {1/1,2/4,3/3,4/5,5/2}{
    % name nodes accordingly
    \node[nodestyle] (p\y) at (p.corner \x){\y};
  }
  % draw edges
  \draw[semithick,black,dashed] (p1) -- (p2);
  \draw[semithick,black,dashed] (p1) -- (p3);
  \draw[semithick,red,->] (p1) -- (p4) node [edgename,midway] {2};
  \draw[semithick,blue,<-] (p1) -- (p5) node [edgename,very near end] {1};
  \draw[semithick,green!60!black,->] (p2) -- (p3) node [edgename,near end] {4};
  \draw[semithick,magenta,<-] (p2) -- (p4) node [edgename,midway] {3};
\end{tikzpicture}
  \caption{\label{fig:g7d}Hamiltonian Path}
\end{subfigure}
\caption{\label{fig:g7}%
A walk (a), cycle (b), Eulerian trail (c) and Hamiltonian path (d) are illustrated.
}
\end{figure}



\begin{definition}{ (Cycle)}
A closed walk of length at least 3 and with no repeated edges besides the first vertex being the same as the last is called a cycle.
A Hamiltonian cycle is a cycle in a graph containing every vertex.
\end{definition}
%
\begin{definition}{(Acyclic Graph)}
A graph that contains no cycles is called acyclic.
\end{definition}
%
\begin{definition}{(Hamiltonian / Eulerian Graph)}
A graph \(G = (V;E)\) is said to be Hamiltonian if it contains a Hamiltonian cycle and Eulerian if it contains an Eulerian tour.
\end{definition}

%
\subsection{Trees and Forests}
%
\begin{definition}{(Connectedness)}
A graph \(G\) is connected if for every pair of vertices \(v_1\) and \(v_2\) in \(V\), \(v_2\) is reachable from \(v_1\).
If \(G\) is a digraph, then \(G\) is connected if its underlying graph is connected.
A graph that is not connected is called disconnected.
\end{definition}

\begin{definition}{(Component)}
Let \(G = (V,E)\) be a graph.
A subgraph \(H\) of \(G\) is a component of \(G\) if $H$ is connected, and
 if \(K\) is a subgraph of \(G\) and \(H\) is a proper subgraph of \(K\), then \(K\) is not connected.
%
The number of components of a graph \(G\) is written \(c(G)\).
\end{definition}
%

\begin{definition}{(Vertex Cut and Cut Vertex)}
Let \(G = (V,E)\) be a graph.
A set \(V_0 \subseteq V\) is a vertex cut if the graph \(G_0\) resulting from deleting vertices \(V_0\) from \(G\) has more components than graph \(G\).
If \(V_0 = \{v\}\) is a vertex cut, then \(v\) is called a cut vertex.
\end{definition}
%
\begin{definition}{(Edge Cut and Cut-Edge)}
Let \(G = (V,E)\) be a graph.
A set \(E_0\subseteq E\) is a edge cut if the graph \(G_0\) resulting from deleting edge \(E_0\) from \(G\) has more components than graph \(G\).
If \(E_0 = \{e\}\)is an edge cut, then \(e\) is called a cut-edge.
\end{definition}
%
\begin{definition}{(Minimal Edge Cut)}
Let \(G = (V,E)\).
An edge cut \(E_0\) of \(G\) is minimal.
if when we remove any edge from \(E_0\) to form \(E_{00}\), the new set \(E_{00}\) is no longer an edge cut.
\end{definition}

\begin{figure}
\centering
\begin{subfigure}{0.49\textwidth}
  \centering
  \pgfdeclarelayer{background layer}
\pgfdeclarelayer{foreground layer}
\pgfsetlayers{background layer,main,foreground layer}
\begin{tikzpicture}[
  nodestyle/.style={%
    draw, shape=circle, fill=white,
    node font = \small,
    blur shadow={shadow blur steps=5}}
]

  \node [nodestyle] (1) at (0,0) {1};
  \node [nodestyle] (2) at ($(1) + (20:2cm)$) {2};
  \node [nodestyle] (3) at ($(1) + (-60:1.5cm)$) {3};
  \node [nodestyle] (4) at ($(3) + (1.5cm, 0)$) {4};
  \node [nodestyle,fill=red!30] (5) at ($(2) + (-30:1.5cm)$) {5};
  \node [nodestyle] (6) at ($(5) + (30:2cm)$) {6};
  \node [nodestyle] (7) at ($(6) + (-45:1.5cm)$) {7};
  \node [nodestyle] (8) at ($(5) + (-45:1.5cm)$) {8};
  \node [nodestyle] (9) at ($(8) + (-10:1cm)$) {9};

  \graph[edge quotes={auto}] {
    (1) -- {(2),(3)};
    (2) -- (4);
    (2) -- [red] (5);
    (3) -- (4);
    (4) -- [red] (5);
    (5) -- {(6),(8)};
    (6) -- {(7),(9)};
    (7) -- (9);
    (8) -- (9);
  };

  \coordinate[above=2mm] (labelpos1) at ($(2)!0.6!(5)$);
  \coordinate[below=3.7mm] (labelpos2) at ($(4)!0.6!(5)$);

  \node (label1) at (labelpos1) {$e_1$};
  \node (label2) at (labelpos2) {$e_2$};

  \coordinate (arrow_tip1) at (5.north);
  % \coordinate (arrow_tip2) at ();
  \draw[{Stealth[round]}-] (arrow_tip1) -- ++ (80:1cm) node [above] {Cut Vertex};
  % \draw[{Stealth[round]}-] (arrow_tip1) -- ++ (-80:1cm) node [below] {Articulation Point};

  \begin{pgfonlayer}{background layer}
    \draw[dashed] plot [smooth cycle] coordinates{
    (2.east)
    (4.north east)
    ($(4.south east)+(-40:0.5cm)$)
    ($(4.south east)+(10:1cm)$)
    (5.south west)
    (5.north west)
    ($(5.north)+(100:5mm)$)
    ($(2.east)+(60:5mm)$)
    };
  \end{pgfonlayer}

  \draw [{Stealth[round]}-]
  ($(4.south east)+(-20:7mm)$)
  -- ++ (-10:5mm) node [right]{Edge Cut};
\end{tikzpicture}


  \caption{\label{fig:g9a}{Edge Cut and Cut Vertex}}

\end{subfigure}
\begin{subfigure}{0.49\textwidth}
  \centering
  \pgfdeclarelayer{background layer}
\pgfdeclarelayer{foreground layer}
\pgfsetlayers{background layer,main,foreground layer}
\begin{tikzpicture}[
  nodestyle/.style={%
    draw, shape=circle, fill=white,
    node font = \small,
    blur shadow={shadow blur steps=5}}
]
  \node [nodestyle] (1) at (0,0) {1};
  \node [nodestyle,fill=red!30] (2) at ($(1) + (20:2cm)$) {2};
  \node [nodestyle] (3) at ($(1) + (-60:1.5cm)$) {3};
  \node [nodestyle,fill=red!30] (4) at ($(3) + (1.5cm, 0)$) {4};
  \node [nodestyle] (5) at ($(2) + (-30:1.5cm)$) {5};
  \node [nodestyle] (6) at ($(5) + (-10:1.5cm)$) {6};
  \node [nodestyle] (7) at ($(6) + (45:1.5cm)$) {7};
  \node [nodestyle] (8) at ($(6) + (-45:1.5cm)$) {8};
  \node [nodestyle] (9) at ($(6) + (2cm,0)$) {9};

  \graph[edges=semithick]{
    (1) -- {(2),(3)};
    (2) -- (4);
    (2) -- (5);
    (3) -- (4);
    (4) -- (5);
    (5) -- [red] (6);
    (6) -- {(7),(8),(9)};
    (7) -- (9);
    (8) -- (9);
  };

  \coordinate[above=2mm] (labelpos) at ($(5)!0.5!(6)$);
  \node (label) at (labelpos) {$e_1$};
  \draw [{Stealth[round]}-] (label) -- ++ (110:1cm) node [above] {Cut Edge};

  \begin{pgfonlayer}{background layer}
    \draw[dashed] plot [smooth cycle] coordinates {
      ($(2.north)+(90:3mm)$)
      ($(2.east)+(45:4mm)$)
      ($(4.north east)+(40:3mm)$)
      ($(4.south east)+(-45:3mm)$)
      ($(4.south west)+(-135:3mm)$)
      ($(2.west) + (180:3mm)$)
    };
  \end{pgfonlayer}

  \draw [{Stealth[round]}-]
  ($(4.south east)+(-45:3.1mm)$)
  -- ++ (-10:5mm) node [right]{Vertex Cut};

\end{tikzpicture}
  \caption{\label{fig:g9b} Vertex Cut and Cut Edge}
\end{subfigure}
\caption{\label{fig:g9}
We illustrate a vertex cut and a cut vertex (a singleton vertex cut) and an edge cut and a cut edge (a singleton edge cut).
Cuts are sets of vertices or edges whose removal from a graph creates a new graph with more components than the original graph.
}
\end{figure}
%
\begin{example}
In Figure~\ref{fig:g9}, we illustrate a vertex cut and a cut vertex (a singleton vertex cut) and an edge cut and a cut edge (a singleton edge cut).
Note that the edge-cut in Figure~\ref{fig:g9a} is minimal and cut-edges are always minimal.
A cut-edge is sometimes called a bridge because it connects two distinct components in a graph.
Bridges (and small edge cuts) are a very important part of social network analysis because they represent connections between different communities.
To see this, suppose that (for example) Figure~\ref{fig:g9b} represents the communications connections between individuals in two terrorist cells.
The fact that Member 5 and 6 communicate and that these are the only two individuals who communicate between these two cells could be important for finding a way to capture or disrupt this small terrorist network.
\end{example}
%
\begin{theorem}
Let \(G = (V,E)\) be a connected graph and let \(e \in E\).
Then \(G_0 = G-\{e\}\) is connected if and only if \(e\) lies on a cycle in \(G\).
\end{theorem}
%
\begin{proof}
Recall a graph \(G\) is connected if and only if for every pair of vertices \(v_1\) and \(v_{n+1}\) there is a walk \(w\) from \(v_1\) to  \( v_{n+1}\) with: \(w = (v_1, e_1, v_2,\dots,v_n, e_n, v_{n+1})\).
Let \(G_0 = G-\{e\}\).
Suppose that \(e\) lies on a cycle \(c\) in \(G\) and choose two vertices \(v_1\) and \(v_{n+1}\) in \(G\).
If \(e\) is not on any walk \(w\) connecting \(v_1\) to \(v_{n+1}\) in \(G\) then the removal of \(e\) does not affect the reachability of \(v_1\) and \(v_{n+1}\) in \(G_0\).
Therefore assume that \(e\) is in the walk \(w\).
The fact that \(e\) is in a cycle of \(G\) implies we have vertices \(u_1,\dots,u_m\) and edges \(f_1,\dots f_M\) so that: \(c = (u_1,f_1,\dots,u_m,f_m,u_1)\) is a cycle and \(e\) is among the \(f_1,\dots,f_m\).
Without loss of generality, assume that \(e = f_m\) and that \(e = \{u_m, u_1\}\) (If otherwise, we can re-order the cyle to make this true).
Then in \(G_0\) we will have the path: \(c_0 = (u_1, f_1,\dots,i_m)\).
The fact that \(e\) is in the walk \(w\) implies there are vertices \(v_i\) and \(v_{i+1}\) so that \(e = \{v_i, v_{i+1}\}\) (with \(v_i = u_1\) and \(v_{i+1} = u_m\)).
In deleting \(e\) from \(G\) we remove the sub-walk \((v_i, e, v_{i+1})\) from \(w\).
But we can create a new walk with structure: \(w_0 = (v_1, e_1,\dots,;\allowbreak v_i, f_1, u_2,\dots,\allowbreak u_{m-1},f_m,u_m,\dots,\allowbreak e_n,v_{n+1})\).
This is illustrated in Figure~\ref{fig:g10}.
\end{proof}
%
\begin{figure}
\centering
\begin{tikzpicture}[
  nodestyle/.style={%
    draw, shape=circle, fill=white, minimum size=0.35cm,
    blur shadow={shadow blur steps=5}}
]
\node[nodestyle] (v1) at (0,0) {};
\node[nodestyle] (v2) at ($(v1) + (1,0)$) {};
\node[nodestyle] (v3) at ($(v2) + (1.5,0)$) {};

\node[%
  regular polygon,
  regular polygon sides=7,
  minimum size=2cm,
] (p) at (4,0.91) {};

\foreach\x in {1,3,4,5,6,7}{
  % name nodes accordingly
  \node[nodestyle] (p\x) at (p.corner \x){};
};

\node[nodestyle] (v4) at ($(p5) + (1.5,0)$) {};
\node[nodestyle] (v5) at ($(v4) + (1,0)$) {};
\node[nodestyle] (v6) at ($(v5) + (1,0)$) {};

\graph[edge quotes={auto}]{
  (v1) -- (v2);
  (v2) -- [dotted, ultra thick] (v3);
  (v3) -- (p4);
  (p3) -- [red,"$u_1$" black] (p4)
       -- [ultra thick, "\large$e$"] (p5)
       -- [red, "$u_m$" black] (p6)
       -- [red] (p7)
       -- [red] (p1)
       -- [dotted, ultra thick] (p3);
  (p5) -- [dotted, ultra thick] (v4);
  (v4) -- (v5); (v5) -- (v6);
};

% Labels
\node at (v1) [below=7pt] {$v_1$};
\node at (v2) [below=7pt] {$v_2$};
\node at (p4) [below=7pt] {$v_i$};
\node at (p5) [below=7pt] {$v_{i+1}$};
\node at (v6) [below=7pt] {$v_{n+1}$};

% Arrow on top
\coordinate[above=4mm] (v1_top) at (v1);
\coordinate[above=4mm] (v3_top) at ($(v3)!0.2!(p4)$);
\coordinate (p3_near) at ($(p3.west)-(4mm,0)$);
\coordinate[above=4mm] (p1_near1) at (p1.west);
\coordinate[above=4mm] (p1_near2) at (p1.east);
\coordinate (p7_near) at ($(p7.east)+(30:3mm)$);
\coordinate (p6_near) at ($(p6.east)+(10:2.5mm)$);
\coordinate[above=4mm] (p5_top) at ($(p5)!0.6!(v4)$);
\coordinate[above=4mm] (v6_top) at (v6.east);

\draw[-{Stealth[round,length=2mm]}] (v1_top) -- (v3_top);
\draw[-{Stealth[round,length=2mm]}]
  (v3_top)
  [rounded corners]-- (p3_near)
  [rounded corners]-- (p1_near1)
  [rounded corners]-- (p1_near2)
  [rounded corners]-- (p7_near)
  [rounded corners]-- (p6_near)
  [rounded corners]-- (p5_top);
\draw[-{Stealth[round,length=2mm]}] (p5_top) -- (v6_top);

\end{tikzpicture}
\caption{\label{fig:g10}%
If \(e\) lies on a cycle, then we can repair path \(w\) by going the long way around the cycle to reach \(v_{n+1}\) from \(v_1\).
}
\end{figure}
%

%
\begin{definition}{ (Forests and Trees)}
Let \(G = (V;E)\) be an acyclic graph.
If \(G\) has more than one component, then \(G\) is called a forest.
If \(G\) has one component, then \(G\) is called a tree.
\end{definition}
%
\begin{figure}
\centering
\begin{tikzpicture}[
  every node/.style={node font=\tiny, shape=circle, fill=yellow},
  level/.style={draw=blue,semithick,sibling distance=25mm}
]
  \node {1}
  child {node {5}
  child {node {8}
    child {node {2}
      child {node {6}
        child {node {4}}
      }
      child {node {10}
        child {node {7}}
        child {node {9}}
      }
    }
    child {node {3}}
  }
  };
\end{tikzpicture}
\caption{\label{fig:g8}%
A tree is shown.
Imagining the tree upside down illustrates the tree like nature of the graph structure.
}
\end{figure}
%
\begin{example}
A randomly generated tree with 10 vertices is shown in Figure~\ref{fig:g8}.
Note that a tree (if drawn upside down) can be made to look exactly like a real tree growing up from the ground.
\end{example}
%
\begin{remark}
We can define directed trees and directed forests as acyclic directed graphs.
Generally speaking, we require the underlying graphs to be acyclic rather than just having no directed cycles.
For the remainder of this section we will deal undirected trees, but results will apply to directed trees unless otherwise noted.
\end{remark}
%
\begin{definition}{ (Spanning Forest)}
Let \(G = (V,E)\) be a graph.
If \(F = (V_0;E_0)\) is an acyclic subgraph of \(G\) such that \(V = V_0\) then \(F\) is called a spanning forest of \(G\).
If \(F\) has exactly one component, then \(F\) is called a spanning tree.
\end{definition}

\begin{theorem}
If \(G = (V,E)\) is a connected graph, then there is a spanning tree \(T =(V,E_0)\) of \(G\).
\end{theorem}

\begin{proof}
We proceed by induction on the number of vertices in \(G\).
If \(|V| = 1\), then \(G\) is itself a (degenerate) tree and thus has a spanning tree.
Now, suppose that the statement is true for all graphs \(G\) with\( |V|= n\).
Consider a graph \(G\) with \(n+1\) vertices.
Choose an arbitrary vertex \(v_{n+1}\) and remove it and all edges of the form \(\{v, v_{n+1}\}\) from \(G\) to form \(G_0\) with vertex set \(V_0 =\{v_1,\dots,v_n\}\).
The graph \(G_0\) has \(n\) vertices and may have \(m\geq 1\) components (\(m > 1\) if \(v_{n+1}\) was a cut-vertex).
By the induction hypothesis, there is a spanning tree for each component of \(G_0\) since each of these components has at most \(n\) vertices.
Let \(T_1,\dots,T_m\) be the spanning trees for these components.
Let \(T_0\) be the acyclic subgraph of \(G\) consisting of all the components spanning trees.
For each spanning tree, choose exactly one edge of the form \(e(i) = \{v_{n+1}, v(i)\}\), where \(v(i)\) is a vertex in component \(i\) and add this edge to \(T_0\).
It is easy to see that no cycle is created in \(T\) through these operations because by construction, each edge \(e(i)\) is a cut-edge and by Corollary 3.43 it cannot lie on a cycle. The graph \(T\) contains every vertex of \(G\) and is connected and acyclic.
Therefore it is a spanning tree of \(G\).
The theorem then follows by induction.
\end{proof}
%
\begin{corollary}
Every graph \(G = (V,E)\) has a spanning forest \(F = (V,E_0)\).
\end{corollary}


% Spectral Graph Theory
\section{Spectral graph theory}
%
\subsection{Representing graphs as matrices}

Instead of storing a graph as a set of vertices and a set of edges we can represent a graph as a matrix. Let \(G\) be a graph with \(n\) nodes.
We denote the vertices by \(v_{1},\ldots,v_{n}\).

\begin{definition}
The adjacency matrix \(A=A\left(G\right) =\left[a_{ij}\right]\) defined to be the \(n\times n\) matrix such that \(a_{ij}=1\) if \(v_{i}v_{j}\in E\left(G\right)\).
This matrix will be symmetric for an undirected graph.
\end{definition}

We can easily consider the generalization to directed graphs and multigraphs.

Note that two isomorphic graphs may have different adjacency matrices.
However, they are related by permutation matrices.

\begin{definition}
A permutation matrix is a matrix gotten from the identity by permuting the columns (i.e., switching some of the columns).
\end{definition}

\begin{lemma}
The graphs \(G\) and \(G^{\prime}\) are isomorphic if and only if their adjacency matrices satisfies the following relation
\[
A=P^{T}A^{\prime}P
\]
for some permutation matrix \(P\).
\end{lemma}

\begin{proof}[Proof (sketch)]
Given isomorphic graphs, the isomorphism gives a permutation of the vertices, which leads to a permutation matrix.
Similarly, the permutation matrix gives an isomorphism.
\end{proof}

Now we see that the adjacency matrix can be used to count \(uv\)-walks.

\begin{theorem}
Let \(A\) be the adjacency matrix of a graph \(G\), where \(V\left(  G\right)=\left\{v_{1},v_{2},\ldots,v_{n}\right\}\).
Then the \(\left(i,j\right)\) entry of \(A^{k}\), where \(k\geq1\), is the number of different \(v_{i}v_{j}\)-walks of length \(k\) in \(G\).
\end{theorem}

\begin{proof}
We induct on \(k\).
Certainly this is true for \(k=1\).
Now suppose \(A^{k}=\left(a_{ij}^{\left(k\right)}\right)\) gives the number of \(v_{i}v_{j}\)-walks of length \(k\).
We can consider the entries of \(A^{k+1}=A^{k}A\).
We have
\[
a_{ij}^{\left(n+1\right)}=\sum_{k=1}^{p}a_{ik}^{\left(n\right)} a_{kj}.
\]
This is the sum of all walks of length \(n\) between \(v_{i}\) and \(v_{k}\) followed by a walk from \(v_{k}\) to \(v_{j}\) of length \(1\).
All walks of length \(n+1\) are generated in this way, and so the theorem is proven.
\end{proof}

Another widely used matrix to represent a graph is the incidence matrix

\begin{definition}
The unoriented incidence matrix (or simply incidence matrix) of an undirected graph is a $n\times  m $ matrix $B$, where $n$ and $m$ are the numbers of vertices and edges respectively, such that
%\begin{equation}
%B_{ij}=\left\{ \begin{matrix}
%1 & \mbox{if vertex } v_i \mbox{ is incident with edge } e_j\\
%0 &  \mbox{ otherwise} \end{matrix}\right.}
%\end{equation}

$$ B_{ij}=\left\{{\begin{array}{rl}\,1&{\text{if vertex }}v_{i}{\text{ is incident with edge }}e_{j},\\0&{\text{otherwise.}}\end{array}}\right. $$
\end{definition}

\subsection{Graph Laplacian}

Recall that an eigenvalue of a matrix \(M\) is a number \(\lambda\) such that there is a vector \(v\) (called the corresponding eigenvector) such that
\[
Mv=\lambda v.
\]
It turns out that symmetric \(n\times n\) matrices have \(n\) eigenvalues.
Since adjacency matrices of two isomorphic graphs are related by permutation matrices as above, and so the set of eigenvalues of \(A\) is an invariant of a graph.

We will actually use the Laplacian matrix instead of the adjacency matrix.
The Laplacian matrix is defined to be
\[
L=A-D,
\]
where, \(D\) is the diagonal matrix whose entries are the degrees of the vertices (called the degree matrix).
The Laplacian matrix is also symmetric, and thus it has a complete set of eigenvalues.
The set of these eigenvalues is called the spectrum of the Laplacian.
Notice the following.

\begin{theorem}
Let \(G\) be a finite graph.
The eigenvalues of the matrix \(L\) are all nonpositive.
Moreover, the constant vector \(\vec{1}=\left(  1,1,1,\ldots,1\right)\) is an eigenvector with eigenvalue zero.
\end{theorem}

\begin{proof}
It is clear that \(\vec{1}\) is an eigenvector with eigenvalue \(0\) since the sum of the entries in each row must be zero.
Now, notice that we can write,
%
\begin{align*}
v^{T}Lv  &  =\sum v_{i}{\left(  Lv\right)}_{i}\\
&  =\sum_{i}v_{i}\sum_{j}L_{ij}v_{j}\\
&  =\sum_{v_{i}v_{j}\in E}v_{i}\left(  v_{j}-v_{i}\right) \\
&  =\frac{1}{2}\left[  \sum_{v_{i}v_{j}\in E}v_{i}\left(  v_{j}-v_{i}\right)
+\sum_{v_{i}v_{j}\in E}v_{j}\left(  v_{i}-v_{j}\right)  \right] \\
&  =-\frac{1}{2}\sum_{v_{i}v_{j}\in E}{\left(  v_{i}-v_{j}\right)}^{2}\leq0.
\end{align*}
%
(The sums over \(i\) are over all vertices, but the sums over \(v_{i}v_{j}\in E\) is the sum over the edges.)
Now note that if \(v\) is an eigenvector of \(L\) with eigenvalue \(\lambda\), then \(Lv=\lambda v\), and,
\[
v^{T}Lv=\lambda v^{T}v=\lambda\sum_{i}v_{i}^{2}.
\]
Thus we have that,
\[
\lambda=\frac{-\frac{1}{2}\sum_{v_{i}v_{j}\in E}\left(v_{i}-v_{j}\right)
^{2}}{\sum_{i}v_{i}^{2}}\leq0.
\]

\end{proof}

\begin{definition}
The eigenvalues of \(-L\) can be arranged \(0=\lambda_{0}\leq\lambda_{1}\leq\lambda_{2}\leq\cdots\leq\lambda_{n-1}\), where \(n\) is the order of the graph, meaning the number of vertices in the graph. 
The collection \(\left(  \lambda_{0},\lambda_{1},\ldots,\lambda_{n-1}\right)\) is called the \emph{spectrum }of the Laplacian.
\end{definition}

\begin{remark}
Sometimes the Laplacian is taken to be \(D^{-1/2}LD^{-1/2}\).
If there are no isolated vertices, these are essentially equivalent.
\end{remark}

\begin{remark}
Note that the Laplacian matrix, much like the adjacency matrix, depends on the ordering of the vertices and must be considered up to conjugation by permutation matrices.
Since eigenvalues are independent of conjugation by permutation matrices, the spectrum is an isomorphism invariant of a graph.
\end{remark}

We will be able to use the eigenvalues to determine some geometric properties of a graph. Recall that \(\lambda_{0}=0\), which means that the matrix \(L\) is singular and its determinant is zero.
Recall the definition of the adjugate of a matrix.

\begin{definition}
If \(M\) is a matrix, the adjugate is the matrix \(M^{\dag}=\left[M_{ij}^{\dag}\right]\) where \(M_{ij}^{\dag}\) is equal to \({\left(-1\right)}^{i+j}\det\left(\hat{M}_{ij}\right)\), where \(\hat{M}_{ij}\) is the matrix with the \(i\)-th row and \(j\)-th column removed.
\end{definition}

The adjugate has the property that
\[
M{\left(M^{\dag}\right)}^{T}=\left(\det M\right)  I,
\]
where \(I\) is the identity matrix.
Applying this to \(L\) (which is symmetric) gives that
\[
LL^{\dag}=0.
\]
Now, the \(n\times n\) matrix \(L\) has rank less than \(n\).
If it is less than or equal to \(n-2\), then all determinants of \(\left(n-1\right)\times\left(n-1\right)\) submatrices are zero, and hence \(L^{\dag}=0\).
If \(L\) has rank \(n-1\), then it has only one zero eigenvalue, which must be \({\left(1,1,\ldots,1\right)}^{T}\).
Since \(LL^{\dag}=0\), all columns of \(L^{\dag}\) must be a multiple of \({\left(1,1,\ldots,1\right)}^{T}\).
But \(L\) is symmetric, so that means that \(L^{\dag}\) must be a multiple of the matrix of all ones. This means that each entry in \(L^{\dag}\) are the same, which motivates the following definition.

\begin{definition}
We define \(t\left(G\right)\) by
\[
t\left(G\right) ={\left(-1\right)}^{i+j}\det\left(-\hat{L}_{ij}\right)
\]
for any \(i\) and \(j\) (it does not matter since all are the same).
\end{definition}

\begin{remark}
It follows that \(t\left(G\right)\) is an integer.
\end{remark}

\begin{lemma}
\(t\left(G\right)=\frac{1}{n}\lambda_{1}\lambda_{2}\cdots\lambda_{p-1}\).
\end{lemma}

\begin{proof}
In general for a matrix \(A\) with eigenvalues \(\lambda_{0},\ldots,\lambda_{p-1}\) we have that,
\[
\sum_{k=0}^{n-1}\frac{\lambda_{0}\lambda_{1}\lambda_{2}\cdots\lambda_{n-1}%
}{\lambda_{k}}=\sum_{i=0}^{n-1}\det\hat{A}_{ii}.
\]
In our case, \(\lambda_{0}=0\) and the right sum is the sum of \(p\) of the same entries, so the result follows.
\end{proof}

\begin{remark}
It follows that \(t\left(G\right)\geq0\).
\end{remark}

Recall that a spanning tree of \(G\) is a subgraph containing all of the vertices of \(G\) and is a tree.

\begin{theorem}
(Matrix Tree Theorem) The number \(t\left(  G\right)  \) is equal to the number of spanning trees of \(G\).
\end{theorem}

\begin{proof}
Omitted, for now.
\end{proof}

%
\begin{corollary}
\(\lambda_{1}\neq0\) if and only if \(G\) is connected.
\end{corollary}

\begin{proof}
\(\lambda_{1}=0\) if and only if \(t\left(G\right)=0\) since \(t\left(G\right)\) is the product of the eigenvalues \(\lambda_{1}\lambda_{2}\cdots\lambda_{n-1}\) and \(\lambda_{1}\) is the minimal eigenvalue after \(\lambda_{0}\).
But \(t\left(G\right)=0\) means that there are no spanning trees, so \(G\) is not connected.
\end{proof}

Now we can consider the different components.

\begin{definition}
The disjoint union of two graphs \(G=G_{1}\sqcup G_{2}\) is the graph gotten by taking \(V\left(G\right)=V\left(G_{1}\right)\sqcup V\left(G_{2}\right)\) and \(E\left(G\right)=E\left(G_{1}\right)\sqcup E\left(G_{2}\right)\) where, \(\sqcup\) is the disjoint union of sets.
\end{definition}

It is not hard to see that if we number the vertices in \(G\) by first numbering the vertices of \(G_{1}\) and then numbering the vertices of \(G_{2}\), that the Laplacian matrix takes the form%
\[
L\left(  G\right)  =\left(
\begin{array}
[c]{cc}%
L\left(G_{1}\right) & 0\\
0 & L\left(G_{2}\right)
\end{array}
\right).
\]
This means that the eigenvalues of \(L\left(G\right)\) are the union of the eigenvalues of \(L\left(G_{1}\right)\) and \(L\left(G_{2}\right)\).
This implies the following.

\begin{corollary}
If \(\lambda_{k}=0\), then there are at least \(k+1\) connected components of \(G\).
\end{corollary}

\begin{proof}
Induct on \(k\).
We already know this is true for \(k=1\).
Suppose \(\lambda_{k}=0\).
We know there must be at least \(k\) components, since \(\lambda_{k}=0\) implies \(\lambda_{k-1}=0\).
We can then write the matrix \(L\left(G\right)\) in the block diagonal form with \(L\left(G_{i}\right)\) along the diagonal for some graphs \(G_{i}\).
Since \(\lambda_{k}=0\), one of these graphs must have \(\lambda_{1}\left(G_{i}\right)=0\).
But that means that there is another connected component, completing the induction.
\end{proof}


%
\end{document}